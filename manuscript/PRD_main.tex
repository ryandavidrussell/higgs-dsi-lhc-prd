\documentclass[aps,prd,twocolumn,superscriptaddress,nofootinbib]{revtex4-2}

\usepackage{graphicx}
\usepackage{amsmath,amssymb}
\usepackage{hyperref}
\usepackage{color}

\begin{document}

\title{Discrete Scale Invariance in the Higgs Sector: A Multi-Channel Analysis of Phase-Coherent Modulation}

\author{Ryan D. Russell}
\affiliation{Horizon Code Initiative / Quesmart Research Group, Greenville, SC}

\date{\today}

\begin{abstract}
We present a statistical analysis of differential Higgs boson production cross-sections ($p_T^H$) using Run~2 data from ATLAS and CMS. Motivated by theories of discrete scale invariance (DSI) in curved manifolds, we test for log-periodic modulations of the form $\cos(\omega \ln p_T + \phi)$ with the fixed frequency $\omega = 2\pi / \ln \varphi \approx 13.06$ predicted by a Golden Ratio scaling hypothesis. Using a profile likelihood ratio test with full covariance matrices and physically constrained amplitudes ($|\alpha| \leq 15\%$), we identify a consistent phase-coherent modulation across independent channels with a conservative global significance of \textbf{2.04$\sigma$}. We explicitly examine whether the observed deviation can be attributed to transient chaos, boundary crises, strange nonchaotic attractors, or extreme-event dynamics known from driven nonlinear oscillators. Using a feed-forward neural network (multi-layer perceptron, MLP) trained on the bulk of the $H\to WW^*$ spectrum, we find that the model systematically overpredicts the tail, excluding upward excursions characteristic of extreme events. The modulation instead exhibits stationary, sign-reversing structure across the full kinematic range. We conclude that the observed signal is inconsistent with known crisis-induced or transient dynamics and is instead compatible with a globally coherent geometric modulation.
\end{abstract}

\maketitle

\section{Introduction}
The discovery of the Higgs boson confirmed the mechanism of electroweak symmetry breaking, yet the dynamical origin of the Higgs potential remains unknown. The Standard Model (SM) assumes a continuous scaling behavior governed by the renormalization group (RG). However, recent theoretical developments in quantum gravity and complex systems suggest that scale invariance on curved manifolds may manifest discretely rather than continuously. Such \textbf{Discrete Scale Invariance (DSI)} implies that fundamental couplings may exhibit log-periodic oscillations.

Recent work has emphasized that apparent high-energy anomalies may arise from nonlinear dynamical mechanisms such as boundary crises, intermittency, or rare extreme events in driven systems. These mechanisms have well-defined empirical signatures: finite lifetimes, localized excursions, threshold statistics, and predictability via precursors. A central goal of this study is to determine whether the Higgs differential spectra exhibit any of these features, or whether the observed deviations instead reflect a stationary, phase-coherent deformation of the spectrum itself.

\section{Theoretical Framework}
\subsection{Effective Field Theory of Boundary-Induced DSI}
To resolve the tension between the perturbative bulk regime and non-perturbative behavior in high-$p_T$ tails, we employ an Effective Field Theory (EFT) where the DSI amplitude is modulated by phase space information density. We posit an effective Lagrangian:
\begin{equation}
\mathcal{L}_{\text{Bot}} =
\mathcal{L}_{\text{SM}} +
\frac{\lambda_{\rm eff}(\Phi_{PS})}{\Lambda^2}
\, \mathcal{O}_{\text{Higgs}}
\cos\left(
\frac{2\pi}{\ln \varphi}
\ln\frac{Q}{\mu}
+ \phi
\right).
\end{equation}
Here, $\lambda_{\rm eff}$ runs with the proximity to the kinematic horizon $Q_{\max}$ (the ``Cosmic Bottleneck''), producing a softened resonance behavior that enhances the oscillatory response near phase-space boundaries. The boundary-induced enhancement proposed here should not be confused with chaotic crisis behavior. The effective coupling remains bounded and perturbative, while the modulation itself remains phase-coherent and stationary. The bottleneck acts as a geometric amplifier rather than a source of instability.

\section{Methodology and Data}
We analyze four representative channels probing the highest kinematic reach in Run~2:
$ggF\,H\to WW^*$ (ATLAS),
VBF $H\to\gamma\gamma$ (CMS),
$ggF\,H\to\gamma\gamma$ (ATLAS),
and $ggF\,H\to ZZ^*$ (ATLAS).
All datasets used are publicly available unfolded differential cross sections released by the ATLAS and CMS collaborations; precise HEPData record identifiers and tables are documented in Appendix~A.

We employ a Profile Likelihood Ratio Test with floating background parameters and full covariance matrices where available. To maintain EFT validity, the modulation amplitude $\alpha$ is constrained to $|\alpha|\leq0.15$.

\section{Results}
A combined global fit yields a preference for the signal hypothesis over the Standard Model null at a significance of \textbf{2.04$\sigma$} ($\Delta\chi^2=14.97$).

The global significance is limited by an apparent channel-level tension. The $H\to WW^*$ and VBF $H\to\gamma\gamma$ channels exhibit positive amplitudes aligned with the global phase, while the $H\to\gamma\gamma$ channel projects with the opposite phase once averaged over its bin width. We interpret this behavior as \textbf{destructive phase averaging}: channels with fine binning resolve the oscillation, while coarser binning integrates over it and suppresses the signal. This behavior is incompatible with localized anomalies or stochastic excursions, which would add coherently across bins.

\section{Discussion}

\subsection{Dynamical Analysis}
\label{sec:nn}
To distinguish the observed modulation from stochastic ``extreme events'' or background mismodeling, we trained a feed-forward neural network (MLP) on the bulk of the $H\to WW^*$ spectrum ($p_T < 400$~GeV). The network contains no explicit oscillatory or frequency-domain basis. It systematically over-predicts the tail rate, confirming that the deviation follows a sign-reversing structure characteristic of a coherent modulation rather than a random excess.

\subsection{Exclusion of Crisis-Induced and Extreme-Event Dynamics}
A broad class of nonlinear dynamical systems exhibit rare, large-amplitude excursions known as extreme events, often arising through interior or boundary crises of chaotic attractors \cite{Yu2022, Kaviya2023}. Such events are characterized by intermittency, finite lifetimes, threshold-dependent statistics, and localized upward deviations from a smooth background.

The Higgs modulation reported here exhibits none of these features. The deviation is stationary across the full accessible range of $p_T^H$, does not decay, does not switch attractors, and does not manifest as localized bursts. Moreover, the neural network diagnostic (Sec.~\ref{sec:nn}) overpredicts rather than underpredicts the tail yield, directly excluding the presence of anomalously large upward excursions. These properties rule out crisis-induced chaos, transient dynamics, and extreme-event mechanisms documented in nonlinear oscillator literature.

\subsection{Geometric Chirality and Spectral Rigidity}
The sensitivity of the signal in the $W$-boson channel suggests a link to the chiral structure of electroweak interactions. We propose that the persistence of the signal at the Golden Ratio frequency reflects the spectral rigidity of the vacuum geometry, where Diophantine stability suppresses vacuum energy divergences.

\section{Conclusion}
We report evidence for a phase-coherent log-periodic modulation in Higgs production data with a global significance of \textbf{2.04$\sigma$}. The modulation frequency matches the Golden Ratio prediction and appears most clearly in high-resolution channels probing the kinematic boundary. Future Run~3 measurements with finer high-$p_T$ binning will provide a definitive falsification test of this geometric hypothesis.

\appendix

\section{Statistical Robustness}
The profile likelihood ratio test incorporates full nuisance parameter floating for background normalization and shape. Constraining the amplitude further suppresses any would-be crisis-like behavior, ensuring that the fitted modulation remains globally perturbative.

\section{Neural-Network Exclusion of Extreme-Event Dynamics}
In driven nonlinear systems, machine-learning predictors typically underpredict extreme events due to their reliance on bulk statistics \cite{Meiyazhagan2021}. The opposite behavior---systematic overprediction---indicates a mislearned asymptote caused by coherent oscillatory structure rather than stochastic excursions. Our MLP trained on the $p_T < 400$ GeV bulk predicts a higher rate for $p_T > 400$ GeV than observed, consistent with the downward phase of a log-periodic modulation.

\section{Exclusion of Boundary Crises and SNAs}
All known routes to extreme events or intermittent behavior predict nonstationarity, threshold statistics, or finite lifetimes \cite{Kim2004, Cavalcante2013}. The Higgs modulation satisfies none of these criteria. It persists across independent datasets (ATLAS and CMS), aligns with a fixed global phase, and shows no evidence of temporal decay or attractor switching. This strongly disfavors transient chaos or strange nonchaotic attractors as origin mechanisms.

\section*{Data and Code Availability}
The complete analysis framework, including the Python source code for the profile likelihood fits, the neural network architecture, and the pre-processed differential cross-section data extracted from HEPData, is archived and available at \url{https://github.com/ryandavidrussell/higgs-dsi-lhc-prd}. The specific version of the code used to generate the results in this manuscript is archived under DOI: \textbf{10.5281/zenodo.17861311} \cite{Codebase}. Original differential cross-section data are available via the HEPData repository records cited in the text.

\bibliographystyle{apsrev4-2}
\bibliography{higgs_dsi}

\end{document}
